\documentclass[twocolumn,11pt]{jarticle} %記事や論文を書く際にjarticleを使用

%%Prefecence
%枠付き文字出力
\usepackage{ascmac}
\usepackage{fancybox}

%複数段組み
\usepackage{multicol}

%四角の枠出力
\usepackage{framed}
%ベクター形式の画像出力
\usepackage[dvipdfmx]{graphicx}
%目次からページに飛べるようにする
\usepackage[dvipdfmx]{hyperref}
\usepackage{pxjahyper}
%ヘッダーとフッターの作成
\usepackage{fancyhdr}
%複数行コメントアウト
\usepackage{comment}
%\pagestyle{fancy} %ヘッダとフッタが設定可能
\renewcommand{\headrulewidth}{3pt}%ヘッダの線の太さ
\renewcommand{\footrulewidth}{1pt}%フッタの線の太さ


%TeXの書式設定
\usepackage{listings, jlisting}
\lstset{
  language={},%言語種類
  basicstyle={\scriptsize\ttfamily},%ソースコードのフォント設定
  identifierstyle={\scriptsize},%英語の書体
  keywordstyle={\scriptsize\ttfamily}, %intやifなどのキーワードの書体
  ndkeywordstyle={\scriptsize},%キーワードその2の書体
  stringstyle={\scriptsize\ttfamily}, %""で囲まれた文字の書体
  frame={tRBl}, %枠の形
  breaklines=true,%行が長くなった際の改行の有無
  xrightmargin=0zw,%右の余白の大きさ
  xleftmargin=3zw,%左の余白の大きさ
  numbers=left,%行番号表示場所
  stepnumber=1,%行番号の増分
  numbersep=1zw,%行番号と本文の間隔
  numberstyle=\ttfamily, %行番号の書体
  framesep=5pt,%frameまでの間隔
  commentstyle={\ttfamily},%コメントアウトのフォント設定
  flexiblecolumns = true,
  classoffset=1,
  showstringspaces=false, %文字列中のスペースをちゃんと" "と表示
  tabsize=4 %タブの間隔文字数(半角)
}
\usepackage{layout}%レイアウト確認用

%Figure 環境中で Table 環境の見出しを表示・カウンタの操作に必要
\makeatletter
\newcommand{\figcaption}[1]{\def\@captype{figure}\caption{#1}}
\newcommand{\tblcaption}[1]{\def\@captype{table}\caption{#1}}
\makeatother


%本文領域を広め(空白箇所マージン領域を小さめ)に設定
\setlength{\oddsidemargin}{-5mm}%文字の左からの間隔(奇数ページ)
\setlength{\evensidemargin}{-5mm}%文字の左からの間隔(偶数ページ)
\setlength{\topmargin}{-13mm}%用紙の上からの間隔
\setlength{\headheight}{12pt}%ヘッダーの高さ
\setlength{\headsep}{25pt}%ヘッダーとボディとの間隔
\setlength{\textwidth}{170mm}%文字出力の横幅を設定
\setlength{\textheight}{240mm}%文字出力の縦幅を設定
\setlength{\marginparwidth}{20pt}%注釈用の余白
\setlength{\footskip}{40pt}%フッターとボディとの間隔

\thispagestyle{empty} %はじめのページにページ番号を振らない

\title{並列分散処理\\ 最終報告書}
\author{155727B:多和田真悟\\ 155734D:箕輪亜加梨\\ 155706J:久場翔悟\\ 155747F:大田喜亮\\ 155732H:新垣優一郎}
\date{2017/7/26}

\begin{document}

%1ページ目の表紙作成
\begin{titlepage}
    \maketitle
	\thispagestyle{empty}
\end{titlepage}

%2ページ目の目次作成
%\tableofcontents %目次を自動作成するコマンド
\setcounter{page}{1} %ページ数を1とする
\pagestyle{plain}
%スタイルファイルは/usr/local/tex/20○○/tex-dist/tex/latex/ のフォルダに適当に入れて sudo mktexlsr で更新を行う。

\newpage
\begin{abstract}
本レポートでは並列分散処理に関する演習の一例としてマンデルブロ集合の描画の速度比較を行った結果をまとめる。また並列化や分散をすることでどれだけ速度が向上したかの検証やOpenMPといった並列計算機の使い方を本実験を通して学ぶ。
\end{abstract}

\section{背景}
\subsection{OpenMPについて}
OpenMPとは並列コンピューティング環境を利用するために用いられる標準化された基盤のことで、並列計算機のなかでは共有メモリ型にあたる。似た計算機の中にはOpenMPIというものがあるが大きな違いは、OpenMPIでは明示的にメッセージの交換をプログラム中に記述しなければいけないが、OpenMPではOpenMPが使えない環境において無視されるディレクティブを挿入することで並列化を行う。このため並列環境と非並列環境でほぼ同一のソースコードを使用できるという利点がある。

しかし、OpenMPはMPIに比べてメモリアクセスのローカリティが低くなる傾向があるため頻繁なメモリアクセスを要するプログラムではMPIのほうが高速な場合が多い\cite{openmp}。

またOpenMPではFortrun,C,C++を対象として並列分散処理を行う。

\section{実験}
本実験ではマンデルブロー集合の描画の速度を並列化前と並列化後で比較する。

\subsection{結果}
\begin{itemize}
    \item 並列化後と並列化前では処理速度に差が生まれ並列化後のほうが高速に処理することができた。
\end{itemize}

\subsection{考察}
\begin{itemize}
    \item OpenMPを用いた並列化は処理速度の向上に有効な手段の一つであることが考えられる。
\end{itemize}

\section{役割分担の詳細とgithubのリンク}
今回は以下のように役割分担を行った。

\begin{table}[h]
    \scriptsize
    \begin{tabular}{|c|c|}
    \hline
    多和田真悟 & 統括と本レポートの作成 \\ \hline
    箕輪亜加梨 & OpenMPによる並列プログラムの作成と実行 \\ \hline
    久場翔悟 & OpenMPによる並列プログラムの作成と実行 \\ \hline
    太田喜亮 & 最終発表スライドの作成とOpenMPに関する背景と考察の調査 \\ \hline
    新垣優一郎 & 最終発表スライドの作成とOpenMPに関する背景と考察の調査 \\ \hline
    \end{tabular}
\end{table}

\begin{thebibliography}{9}
	\bibitem{openmp} OpenMP --- https://ja.wikipedia.org/wiki/OpenMP
\end{thebibliography}

\end{document}

